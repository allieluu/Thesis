% -*- program: xelatex -*-
\documentclass{article}
\pagenumbering{arabic}

\widowpenalty=9999
\usepackage[normalem]{ulem}
\usepackage{setspace}

\usepackage{fontspec}
\setmainfont{Palatino}

\begin{document}

\title{Machine Learning-based Genome Wide Association Studies of Rheumatoid Arthritis}

\author{Allie Burton\\Advisor: Prof.\ Sandra Batista}

\date{}
\maketitle

\doublespacing

\section*{Introduction}
Scoliosis is a disease marked by curvature of the spine. The most common type,
adolescent idiopathic scoliosis (AIS), generally occurs right before puberty and
has no known causes. According to the Scoliosis Research Society, approximately
30\% of all AIS patients have some family history of scoliosis, so many researchers
now are looking for a genetic component\cite{ScoliosisResearchSociety}. To look
for these genetic components, researchers perform what is known as a genome-wide
association study, or GWAS, which the National Human Genome Research Institute
defines as an ``approach that involves rapidly scanning markers across the
complete sets of DNA, or genomes, of many people to find genetic variations
associated with a particular disease''\cite{NationalHumanGenomeResearchInstitute2015}. 
The results of these studies, however, have been generally inconclusive collectively. 
For example, Takahashi et.\ al.'s\cite{Takahashi2011} GWAS studying approximately 
1,400 Japanese females shared data on 87 of the top 100 single nucleotide 
polymorphisms (SNPs) found in Sharma et.\ al.'s\cite{Sharma2011} racially diverse 
GWAS of 419 families, however, only one of those SNPs showed significant 
association in Takahashi et.\ al's GWAS.\ The goal of my project is to study the 
usage and efficacy of machine learning for GWAS in scoliosis.

\section*{Background}

\subsection*{Related Work}
Although there have not been any studies done to date using machine learning for
GWAS of idiopathic scoliosis, there have been many studies using machine learning
for other phenotypes including IgM and rheumatoid arthritis, as mentioned above,
in addition to myocardial infarction, coronary artery calcification, and anti-cyclic
citrullinated peptide\cite{Szymczak2016}. These studies will provide the basis
for my methodology, specifically D’Angelo et. al.’s\cite{DAngelo2009} and Tang
et.\ al.’s\cite{Tang2009} GWASs of rheumatoid arthritis and Stassen et. al.’s
GWASs of IgM. Since there are no prior machine learning-based GWASs of AIS, I
will replicate their respective methodologies as best as possible, adapting where
necessary for the specifics of scoliosis and the data sets I am using.

\subsection*{Important Terminology}
To conduct a genome-wide association study, researchers get DNA samples from two
groups of people: those with the trait in question and those without it. Using 
these samples, each person's entire genome is scanned in a machine looking for 
single-nucleotide polymorphisms. A single-nucleotide polymorphism, or SNP 
(prounounced ``snip''), is a variation at a single position in an individual's 
DNA sequence that occurs in one percent or less of the population. If certain 
SNPs occur more frequently in persons with the disease than without it, then 
those SNPs are associated with the trait. Although these SNPs can point to places 
in the human genome that might be related to the source of the trait, the SNPs 
themselves may not be the cause of the trait itself, so researchers often look 
at base pairs in the region to see if they are also related
\section*{Methods}
\subsection{Phase 1}



\bibliographystyle{ieeetr}
\bibliography{Thesis}

\end{document}
